\chapter{Comparación técnica}

La tabla siguiente presenta una comparación técnica de varios sistemas operativos destacados, considerando aspectos clave como arquitectura, lenguaje de programación, tamaño del kernel, soporte de hardware, documentación y comunidad de usuarios y desarrolladores.

\begin{sidewaystable}[p]
\centering
\caption{Comparación resumida de distintos sistemas operativos en orden de popularidad, elaborada en base a la investigación realizada en este proyecto.}
\vspace{0.5cm} 
\begin{tabular}{|c|c|c|c|c|c|c|}
    \hline
    \textbf{Sistema} & \textbf{Arquitectura} & \textbf{Lenguaje} & \textbf{Kernel} & \textbf{HW} & \textbf{Documentación} & \textbf{Comunidad} \\ 
    \hline
    Windows & Monolítica & C++ & Grande & Amplio & Excelente & Masiva \\ 
    \hline
    XNU & Híbrida & C++ & Grande & Intel & Buena & Apple \\ 
    \hline
    Linux & Monolítica & C & Grande & Amplio & Excelente & Activa \\ 
    \hline
    MINIX & Microkernel & C & Pequeño & Limitado & Buena & Académica \\ 
    \hline
    Redox & Microkernel & Rust & Mediano & Limitado & Moderada & Experimental \\ 
    \hline
    xv6 & Monolítica & C & Pequeño & Limitado & Excelente & Académica \\ 
    \hline
    FreeRTOS & Microkernel & C & Minúsculo & Amplio & Extensa & Embebidos \\ 
    \hline
    Fuchsia & Microkernel & C++ & Mediano & IoT & Buena & Google \\ 
    \hline
    Haiku & Híbrida & C++ & Mediano & Limitado & Buena & Constante \\ 
    \hline
    ToaruOS & Híbrida & C & Pequeño & Limitado & Básica & Reducida \\ 
    \hline
\end{tabular}
\label{tab:comparacion_so_custom_h}
\end{sidewaystable}
